\documentclass[a4paper,12pt]{article}
\usepackage[utf8]{inputenc}

\title{Distributed Algorithms Lab Assignment 3: \\ Experimental results}
\author{Roshan Timal \\ 4030087 \and Jasper Ruoff \\ 4003055}

\begin{document}

\maketitle

\section*{Introduction}
In this report we will experimentally evaluate the algorithm described by Lamport-Pease-Shostak for Byzantine Agreement.

\section{Message complexity}
The first experiment we have conducted studies how many messages $m$ are circulating in the distributed system with varying numbers of processes $p_n$ (8, 16, 24, 32, 40 and 48) and with a fixed amount of faulty processes $p_f = 2$. The results are shown in the table below.

%$f=2$, $p_f=2$, static, excl com

\begin{table}[!ht]
	\centering
	\begin{tabular}{r|r}
		$p_n$ & $m$ \\ \hline
		8 & 260 \\
		16 & 2956 \\
		24 & 11156 \\
		32 & 27932 \\
		40 & 56356 \\
		48 & 99500
	\end{tabular}
\end{table}

The number of messages $m$ grows polynomially with the number of processes $p_n$.

The second experiment also measures the number of messages $m$ send in the distributed system. However it will be conducted with a fixed number of processes $p_n = 16$ and a varying number of faulty processes $p_f$ (1, 2, 3, 4 and 5). The results are shown in the table below.

%$p_n=16$, static, excl com

\begin{table}[!ht]
	\centering
	\begin{tabular}{r|r}
		$f = p_f$ & $m$ \\ \hline
		1 & 226 \\
		2 & 2956 \\
		3 & 35716 \\
		4 & 396076
	\end{tabular}
\end{table}

The number of messages $m$ grows exponentially with the number of faults $f$ to be detected.

\section{System behavior}
The next experiment will focus more on the behavior of faulty processes. 
While we assumed in the first two experiments that a faulty process would always give a faulty answer, the faulty processes in the third experiment will exhibit random behavior, a faulty process propagates a wrong value with a 50\% chance. 
The third experiment will be executed with a fixed amount of processes $p_n = 8$ and varying number of faulty processes $p_f$ (3, 4, 5 and 6). 
The commander, the process initiating the OM function, will be excluded from being a faulty process, as was also the case for the first and the second experiment.

Because the experiment introduces a random behavior, we have repeated each experiment 10 times, and calculated the average. The table below shows the percentage of processes that got a wrong order.

%$p_n=8$, $f=2$, random, excl com, 10 rounds

\begin{table}[!ht]
	\centering
	\begin{tabular}{r|r}
		$p_f$ & Wrong order (\%) \\ \hline
		3 & 7.5 \\
		4 & 11 \\
		5 & 44 \\
		6 & 54
	\end{tabular}
\end{table}

It is clearly visible that as the number of faulty processes $p_f$ increases, more processes get a wrong order.

When the commander is a traitor, it is very rare for the processes to reach agreement if the number of faulty processes exceeds the maximum. Furthermore, in that case, it follows naturally that 50\% of the processes gets a wrong order on average, when we assume the random fault behavior as described above.

\end{document}
